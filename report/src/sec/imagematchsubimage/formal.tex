\section{Análise formal}
\label{sec:imagematchsubimage/formal}

\subsection{Descrição do algoritmo}

A função ImageMatchSubImage compara uma imagem (img2) com uma subimagem de uma imagem maior (img1). A subimagem é definida pela posição inicial (x, y) na imagem maior. Após as verificações iniciais, percorre-se cada pixel da imagem, a partir da posição inicial, comparando-o com o da subimagem. No caso de encontrar um pixel que não corresponda, a função devolve falso. Se todos os pixeis corresponderem, a função devolve verdadeiro.

\subsection{Análise de complexidade}

\begin{itemize}
    
\item
\textbf{Complexidade Temporal}

Após verificarmos que a obtenção de pixeis, presente dentro do loop interior, é uma operação $O(1)$, e que se percorre a imagem linha a linha e coluna a coluna, podemos afirmar que a complexidade temporal do algoritmo é $O(n*m)$, onde n e m são a altura e largura de img2, respetivamente.

\item
\textbf{Complexidade Espacial}

A complexidade espacial do algoritmo é $O(1)$, dado que não são usadas estruturas de dados adicionais que cresçam com o tamanho da entrada. As variáveis row e col são as únicas variáveis adicionais, e ocupam um espaço constante.

\end{itemize}
