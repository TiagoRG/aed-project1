\section{Análise formal}
\label{sec:imagelocatesubimage/formal}

\subsection{Descrição do algoritmo}

A função ImageLocateSubImage compara uma imagem (img2) com uma imagem maior (img1). Após as verificações iniciais, percorre-se cada pixel da imagem, comparando-o com o da subimagem. Se for detetado um pixel igual, chama-se a função ImageMatchSubImage para verificar se a subimagem está presente na imagem maior. No caso de encontrar um pixel que não corresponda, a função continua até encontrar, ou até acabar a imagem. Caso a ImageMatchSubImage devolva verdadeiro, a função devolve verdadeiro.

\subsection{Análise de complexidade}

\begin{itemize}
    
\item
\textbf{Complexidade Temporal}

Sabemos que ImageMatchSubImage, presente dentro do loop interior, é uma operação $O(p*o)$, e que se percorre a imagem linha a linha e coluna a coluna, logo é possível afirmar que a complexidade temporal do algoritmo é $O((n*m)*(p*o))$, onde n e m são a altura e largura de img2 e p e o são a altura e largura da subimagem verificada dentro do loop, respetivamente.

\item
\textbf{Complexidade Espacial}

A complexidade espacial do algoritmo é $O(1)$, dado que não são usadas estruturas de dados adicionais que cresçam com o tamanho da entrada. As variáveis row e col são as únicas variáveis adicionais, e ocupam um espaço constante.

\end{itemize}
