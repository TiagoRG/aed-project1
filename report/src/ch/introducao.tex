%! Author = TiagoRG
%! GitHub = https://github.com/TiagoRG

\chapter{Introdução}
\label{ch:introducao}
{
%%%
% Conteúdo da introdução aqui

Neste projeto será desenvolvida uma ferramenta de manipulação de imagens monocromáticas (apenas um canal de cor, neste caso apenas escala de cinzentos), que permitirá efetuar diversas operações:

\begin{itemize}
    \item \textbf{ImageStats:} Mostra os valores mínimo e máximo dos pixeis da imagem;
    \item \textbf{ImageNegative:} Inverte os pixeis da imagem ($new\_value = max\_value - old\_value$);
    \item \textbf{ImageThreshold:} Aplica um limiar à imagem, de forma a que os pixeis com valor inferior ao limiar sejam convertidos para 0 e os restantes para o valor máximo;
    \item \textbf{ImageBrighten:} Aumenta ou diminui o brilho da imagem, multiplicando todos os pixeis por um fator ($fator~>~1$ aumenta o brilho, $fator~<~1$ diminui o brilho);
    \item \textbf{ImageMirror:} Espelha a imagem, horizontalmente;
    \item \textbf{ImageRotate:} Roda a imagem, 90º no sentido contrário ao dos ponteiros do relógio;
    \item \textbf{ImageCrop:} Corta a imagem, de acordo com as coordenadas do canto superior esquerdo e o tamanho do retângulo a cortar;
    \item \textbf{ImagePaste:} Cola uma imagem sobre outra a partir das coordenadas do canto superior esquerdo;
    \item \textbf{ImageBlend:} Cola uma imagem sobre outra, com um fator de transparência (entre 0 e 1), a partir das coordenadas do canto superior esquerdo;
    \item \textbf{ImageMatchSubImage:} Verifica se uma imagem está contida dentro de outra maior, a partir das coordenadas do canto superior esquerdo;
    \item \textbf{ImageLocateSubImage:} Procura uma imagem dentro de outra maior e devolve as coordenadas do canto superior esquerdo da primeira ocorrência;
    \item \textbf{ImageBlur:} Aplica um filtro de desfoque à imagem;
\end{itemize}

Tudo isto será desenvolvido usando a linguagem de programação C com recurso às bibliotecas stdio.h, stdlib.h, assert.h, ctype.h e errno.h.

\vspace{0.5cm}

Para as funções ImageLocateSubImage e ImageBlur será ainda feita a análise formal onde será apresentada a complexidade temporal e espacial de cada uma das funções. Para tal, será usada a notação O-grande, onde se considera apenas o termo de maior ordem. Será feita também uma análise experimental destas duas funções para verificar o resultado expectado pela análise formal. Por último, será descrito todo o algoritmo otimizado da função ImageBlur.

%%%
}
